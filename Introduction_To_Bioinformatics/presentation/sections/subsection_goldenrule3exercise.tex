% Third Golden Rule of Bioinformatics Exercise
%
% The aim of this exercise is to demonstrate that a simple question 
% of biological relevance does not have the obvious answer. The point
% is to reinforce that even "simple" questions may require sufficient 
% thought and attention, in order to answer them properly. Also, that 
% "intuition" is not always sufficient for even "simple" questions.

\subsection{Golden Rule 3 Exercise}
\begin{frame}
  \frametitle{Exercise 3}
  \framesubtitle{Classification}
  \begin{itemize}
    \item Rule: If there is a vowel on one side of the card, there \textit{must} be an even number on the other side.
    \item Which cards \textit{must} be turned over to determine if this rule (if a card shows a vowel on one face, the opposite face is even) holds true?
  \end{itemize}
  \includegraphics[width=0.8\textwidth]{images/wason}
\end{frame}

\begin{frame}
  \frametitle{Exercise 3}
  \framesubtitle{Classification}
  This is the Wason Selection Task
  \begin{itemize}
    \item<1-> If you chose \emph{E} and \emph{4}
    \begin{itemize}
      \item<2-> You are in the typical majority group
      \item<2-> You are not correct
      \item<2-> You have been a victim of confirmation bias (System 1 thinking)
    \end{itemize}
    \item<3-> If you chose \emph{E} and \emph{7}
    \begin{itemize}
      \item<4-> Congratulations!
      \item<4-> Your choice was capable of \textit{falsifying} the rule.
    \end{itemize}
  \end{itemize}
\end{frame}

\begin{frame}
  \frametitle{Exercise 3}
  \framesubtitle{Classification}
  Rule: If there is a vowel on one side of the card, there \textit{must} be an even number on the other side.    
  \begin{center}
  \begin{tabular}{c|c|c}
	  Card & Outcome & Rule \\
	  \hline
	  \hline
	    \multirow{2}{*}{E} & Even & Can be true even if rule false \\
	                                & Odd & \emph{violated} \\
	  \hline
	    \multirow{2}{*}{K} & Even & na \\
	                                & Odd & na \\	    
	  \hline
	    \multirow{2}{*}{4} & Vowel & Can be true even if rule false \\
	                                & Consonant & na \\
	  \hline
	    \multirow{2}{*}{7} & Vowel & \emph{violated} \\
	                                & Consonant & na \\	    
  \end{tabular}
  \end{center}
\end{frame}

\begin{frame}
  \frametitle{Exercise 3}
  \framesubtitle{Classification}
  \begin{itemize}
    \item This is equivalent to functional classification, e.g:
    \item Rule: If there is a CRN/RxLR/T3SS domain, the protein \textit{must} be an effector.
  \end{itemize}
  \includegraphics[width=\textwidth]{images/wason_rxlr}
\end{frame}
  
\begin{frame}
  \frametitle{Exercise 3}
  \framesubtitle{Classification}
  \begin{itemize}
    \item Confirmation Bias (Wason Selection Task)
    \begin{itemize}
      \item An uninformative experiment is performed
      \item \url{http://en.wikipedia.org/wiki/Wason_selection_task}
    \end{itemize}
    \item Affirming the Consequent (a related formal fallacy)
    \begin{enumerate}
     \item If $P$, then $Q$
     \item $Q$
     \item Therefore, $P$
    \end{enumerate}
    \begin{itemize}
      \item Experimental results are misinterpreted
      \item \url{http://en.wikipedia.org/wiki/Affirming_the_consequent}
    \end{itemize}
  \end{itemize}
\end{frame}
