% Which BLAST tool should I use?
%
% Short introduction to some of the range of BLAST tools available

\subsection{Which BLAST tool should I use?}
\begin{frame}
  \frametitle{Multiple BLAST \textit{tools}}
  \begin{itemize}
    \item BLASTN \textit{vs} MEGABLAST \textit{vs} TBLASTX \textit{vs} ...?
    \item Korf \textit{et al.} (2003) BLAST is really good for theory part, \\
             but practical examples dated due to changes with BLAST+
  \end{itemize}
  \begin{center}
    \includegraphics[width=.2\textwidth]{images/korf_book}
  \end{center}
\end{frame}

\begin{frame}
  \frametitle{Multiple \textit{ways} to run BLAST}
  \begin{itemize}
    \item BLAST+ at the command line (today)
    \item Via a script or programming language
    \item Via a graphical tool like BioEdit, CLCbio, Blast2GO
    \item Via the NCBI website
    \item Via a genome consortium website
    \item Via a Galaxy web server
    \item etc
    \item Offers flexibility \textit{but} different settings/options/versions
  \end{itemize}
\end{frame}

\begin{frame}
  \frametitle{Multiple \textit{places} to run BLAST}
  \begin{itemize}
    \item On the NCBI servers, e.g. via website or tool
    \item On 3rd party servers, e.g. via websites
    \item On your own computer
    \item On our Linux cluster
  \end{itemize}
\end{frame}

\begin{frame}
  \frametitle{Core BLAST tools: Query sequences \textit{vs} Database}
  \begin{itemize}
    \item Nucleotide \textit{vs} Nucleotide:
    \begin{itemize}
      \item \texttt{blastn} (covering blastn, megablast, dc-megablast)
    \end{itemize}
    \item Translated nucleotide \textit{vs} Protein:
    \begin{itemize}
      \item \texttt{blastx}
    \end{itemize}
    \item Protein \textit{vs} Translated nucleotide:
    \begin{itemize}
      \item \texttt{tblastn}
    \end{itemize}
    \item Protein \textit{vs} Protein:
    \begin{itemize}
      \item \texttt{blastp}, \texttt{psiblast}, \texttt{phiblast}, \texttt{deltablast}
    \end{itemize}
  \end{itemize}
  See \url{http://blast.ncbi.nlm.nih.gov/} for a reminder ;)
\end{frame}
