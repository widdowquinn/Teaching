\documentclass[table]{beamer}
\mode<presentation>
\usetheme{Berlin}
\usecolortheme{beaver}
\usepackage{listings}
\usepackage{multirow}
\usepackage{xcolor}

%%%
% LISTINGS SETTING
\lstset{ %
  backgroundcolor=\color{yellow},   % choose the background color; you must add \usepackage{color} or \usepackage{xcolor}
  basicstyle=\tiny\ttfamily,        % the size of the fonts that are used for the code
  breakatwhitespace=false,         % sets if automatic breaks should only happen at whitespace
  breaklines=true,                 % sets automatic line breaking
  captionpos=b,                    % sets the caption-position to bottom
  commentstyle=\color{red},    % comment style
  deletekeywords={...},            % if you want to delete keywords from the given language
  escapeinside={\%*}{*)},          % if you want to add LaTeX within your code
  extendedchars=true,              % lets you use non-ASCII characters; for 8-bits encodings only, does not work with UTF-8
  frame=single,                    % adds a frame around the code
  keepspaces=true,                 % keeps spaces in text, useful for keeping indentation of code (possibly needs columns=flexible)
  keywordstyle=\color{blue},       % keyword style
%  language=Octave,                 % the language of the code
  morekeywords={*,...},            % if you want to add more keywords to the set
  numbers=left,                    % where to put the line-numbers; possible values are (none, left, right)
  numbersep=5pt,                   % how far the line-numbers are from the code
  numberstyle=\tiny\color{gray}, % the style that is used for the line-numbers
  rulecolor=\color{black},         % if not set, the frame-color may be changed on line-breaks within not-black text (e.g. comments (green here))
  showspaces=false,                % show spaces everywhere adding particular underscores; it overrides 'showstringspaces'
  showstringspaces=false,          % underline spaces within strings only
  showtabs=false,                  % show tabs within strings adding particular underscores
  stepnumber=1,                    % the step between two line-numbers. If it's 1, each line will be numbered
  stringstyle=\color{mauve},     % string literal style
  tabsize=4,                       % sets default tabsize to 2 spaces
  title=\lstname                   % show the filename of files included with \lstinputlisting; also try caption instead of title
}


%%%
% TITLE PREAMBLE
\title[Intro to Bioinformatics] % (optional, only for long titles)
{An Introduction to Bioinformatics Tools}
\subtitle{Part 3: Workshop}
\author[Pritchard, Cock] % (optional, for multiple authors)
{Leighton~Pritchard \and Peter~Cock}
\institute[The James Hutton Institute] % (optional)
{
  Information and Computational Sciences\\
  The James Hutton Institute
}
\date[May 2014] % (optional)
{Bioinformatics Training, 29$^{th}$,30$^{th}$ May 2014}
\subject{Bioinformatics}

%%%
% TOC
% Show table of contents, with current section highlighted,
% at the start of each section
\AtBeginSection[]
{
  \begin{frame}
    \frametitle{Table of Contents}
    \tableofcontents[currentsection,hideothersubsections]
  \end{frame}
}


%%%
% START DOCUMENT
\begin{document}

  \frame[plain]{\titlepage}
  
%%%
% SECTION: Workshop Outline
  \section{Outline}
 
    % What you will be doing
    \begin{frame}
     \frametitle{What You Will Be Doing}
     \begin{itemize}
       \item Functional annotation of a draft bacterial genome
       \begin{itemize}
         \item Genome comparisons
         \item Gene prediction/ORF detection
         \item Gene functional annotation
         \item Gene comparisons
       \end{itemize}
     \end{itemize}
    \end{frame}
    
    \begin{frame}
     \frametitle{Genome comparisons}
     \begin{itemize}
       \item Artemis
       \item BLAST
       \item MUMmer
       \item MAUVE
       \item ACT
       \item Differences between genome aligners
     \end{itemize}
    \end{frame}

    \begin{frame}
     \frametitle{Gene (coding sequence) Prediction}
     \begin{itemize}
       \item ORF detection
       \item GeneMark
       \item Glimmer
       \item Prodigal
       \item Benchmarking gene prediction
     \end{itemize}
    \end{frame}

    \begin{frame}
     \frametitle{Functional Prediction}
     \begin{itemize}
       \item InterPro
       \item RAST
       \item KAAS
       \item PFam
       \item Artemis
       \item Functional classification
     \end{itemize}
    \end{frame}
    
    \begin{frame}
     \frametitle{Gene Comparisons}
     \begin{itemize}
       \item BLAST
       \item HMMer
       \item CLUSTAL
       \item T-COFFEE
       \item JalView
       \item Differences between gene aligners
     \end{itemize}
    \end{frame}


  % Getting started
  \section{Setting Up}
  \begin{frame}
    \frametitle{Locate your data}
    \begin{itemize}
      \item You are in group A, B, C or D - this decides your chromosome sequence: \\
      \texttt{chrA.fasta}, \texttt{chrB.fasta}, \texttt{chrC.fasta}, \texttt{chrD.fasta}
      \item Each sequence represents a single stitched, ordered draft bacterial genome comprising a number of contigs.
      \item You will use your sequence as the basis of the exercises in the workshop.
    \end{itemize}
  \end{frame}  
  
  \begin{frame}
    \frametitle{Locate your data}
    \begin{itemize}
      \item You are in group A, B, C or D - this decides your dataset: \\
      \texttt{chrA.fasta}, \texttt{chrB.fasta}, \texttt{chrC.fasta}, \texttt{chrD.fasta}
      \item You also have a GFF file describing the location of assembled contains \\
      \texttt{chrA\_contigs.gff}, \texttt{chrB\_contigs.gff}, \texttt{chrC\_contigs.gff}, \texttt{chrD\_contigs.gff}
      \item Your data is in \texttt{data/workshop/chromosomes}
    \end{itemize}
  \end{frame}    
  
% [fragile] frames must end with \end{frame} directly following a newline, or they break!
  \begin{frame}[fragile]
    \frametitle{Inspect the data}
    \begin{lstlisting}[language=bash]
$ cd ../../data/workshop/chromosomes
$ head -n 3 chrA.fasta 
>chrA
ttttcttgattgaccttgttcgagtggagtccgccgtgtcactttcgctttggcagcagt
gtcttgcccgtttgcaggatgagttacctgccacagaattcagtatgtggatacgcccgt
$ head -n 3 chrA_contigs.gff 
##gff-version 3
chrA	stitching	contig	1	154993	.	.	.	ID=contig00005_b;Name=contig00005_b
chrA	stitching	contig	155036	241491	.	.	.	ID=contig00018;Name=contig00018
    \end{lstlisting}
\end{frame}  

  % Artemis
  \subsection{Artemis}
% [fragile] frames must end with \end{frame} directly following a newline, or they break!
  \begin{frame}[fragile]
    \frametitle{Inspect the data}
    Starting \texttt{Artemis}
    \begin{lstlisting}[language=bash]
$ art &
    \end{lstlisting}
    \begin{center}
      \includegraphics[width=0.6\textwidth]{images/artemis_splash} 
    \end{center}
\end{frame} 
    
  \begin{frame}
    \frametitle{Load the chromosome sequence}
    \begin{center}
      \includegraphics[width=0.75\textwidth]{images/artemis_files} 
    \end{center}
\end{frame}     
    
  \begin{frame}
    \frametitle{Load the chromosome sequence}
    \begin{center}
      \includegraphics[width=0.8\textwidth]{images/artemis_loaded_seq} 
    \end{center}
\end{frame}     

  \begin{frame}
    \frametitle{Load the contig GFF}
    \begin{center}
      \includegraphics[width=0.3\textwidth]{images/artemis_read_entry} 
    \end{center}
\end{frame}     

  \begin{frame}
    \frametitle{Load the contig GFF}
    \begin{center}
      \includegraphics[width=0.75\textwidth]{images/artemis_select_contig_gff} 
    \end{center}
\end{frame}     

  \begin{frame}
    \frametitle{Load the contig GFF}
    \begin{center}
      \includegraphics[width=0.9\textwidth]{images/artemis_loaded_contigs} 
    \end{center}
\end{frame}     

  \begin{frame}
    \frametitle{Find the stitching sequence}
    The contigs are stitched with a specific sequence: see if you can find, and identify it.
    \begin{center}
      \includegraphics[width=\textwidth]{images/artemis_stitch_tease} 
    \end{center}
\end{frame}     
    
    
% etc
\end{document}